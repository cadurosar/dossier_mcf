
\begin{center}
\begin{tabular}{|c|c|c|c|}
\hline
\textbf{School} & \textbf{Level} & \textbf{Nb. of hours} & \textbf{Section} \\ \hline \hline
{ENSEIRB-MATMECA} & {2015 - 2016} & {45 h} & 61 \\ \hline
{ENSEIRB-MATMECA} & {2018 - 2019} & {192 h} & 61 \\
\hline
\multicolumn{2}{c|}{} & 237 h \\
\cline{3-3}
\end{tabular}
\end{center}
%\vspace{1em}



\subsubsection{Electronics Design Project}
\label{subsubsec:loto}
\begin{description}\parskip 0pt
\item[Person in charge:] Mathieu L�onardon
\item[Level :] First year ENSEIRB - equivalent L3
\item[Volume :] 2 x 25 Hours
\item[Content :]  
The first part of this module is a course on FPGA design flow.
The state machines are also presented and some exercises are proposed.
This is followed by a project to design a digital architecture. The objectives are:
    \begin{itemize}
        \item design a hardware architecture,
        \item acquire VHDL skills,
        \item develop a synthetic mind for the writing of the report,
        \item acquire skills in project presentation with a defense.
    \end{itemize}
    
The project is the realization of a Lotto on a Nexys 4 card. The end user must be able to draw 6 numbers at random, without removing them. These numbers are displayed on 7-segment displays.
\item[Keywords:] FPGA, VHDL, Digital Electronics 
\item[Personal participation:] Integrated course, project supervision. 
\item[Educational resources created :] \url{https://bit.ly/2HLfaXt}
\end{description}

\subsubsection{Reconfigurable architecture}
\label{subsubsec:reconf}
\begin{description}\parskip 0pt
\item[Person in charge:] Mathieu L�onardon
%\item[Enrolment:] About 15 students
\item[Level :] Second year ENSEIRB-MATMECA - equivalent M1
\item[Volume :] 1 x 20 Hours
\item[Content:] Advanced design on FPGA circuits. Objectives :
    \begin{itemize}
        \item define reconfigurable architectures,
        \item understand their internal structure,
        \item understand how they work,
        \item implementation of advanced techniques,
        \item acquire skills in project presentation with a small defense.
    \end{itemize}
\item[Keywords:] FPGA, VHDL, Digital Electronics 
\item[Personal participation:] Integrated course, project management.
\item[Educational resources created :] \url{https://bit.ly/2U4Ua4o }
\end{description}

\subsubsection{Digital Electronics}
\label{subsubsec:rsi}
\begin{description}\parskip 0pt
\item[Person in charge:] Mathieu L�onardon
\item[Level :] First year ENSEIRB-MATMECA - equivalent L3
\item[Volume :] 1 x 25 Hours
\item[Content:] This module introduces the basics of digital electronics: numeration, Boolean algebra, combinatorial logic as well as sequential logic. Architecture and operation of a finite state machine. Basics of printed circuit board technology.
\item[Keywords:] Numbering, Finite state machine, CMOS technology, Digital electronics 
\item[Personal Participation:] Integrated Course - Supervised works
\item[Educational resources created :] \url{https://bit.ly/2U0hgsV}
\end{description}


\subsubsection{Combinatory logic and sequential logic}
\label{subsubsec:en1}
\begin{description}\parskip 0pt
\item[Person in charge:] Christophe J�go
\item[Level :] First year ENSEIRB-MATMECA - equivalent L3
\item[Volume :] 1 x 32 Hours
\item[Content :] ~

\begin{itemize}
\item The elementary combinatorial and sequential functions used in digital circuits,
\item the modeling of numerical functions using the VHDL language.
\end{itemize}
At the end of the course, the student must be able to:
\begin{itemize} 
\item to describe a combinatorial function and represent it with a digital circuit,
\item to describe and synthesize a counter, a state machine,
\item to identify the critical path of a complex logical function and calculate its maximum operating frequency.
\end{itemize}
Six four-hour tutorial sessions of four full hours each. Each session is divided into two parts. 
Themes are dealt with in the first part. 
In a second part, the defined digital systems are described in the hardware description language VHDL. 
This approach gradually introduces students to this language. 
\item[Keywords:] Digital electronics, VHDL, FPGA
\item[Personal participation:] Supervision of the practical works and the project
\end{description}

\noindent \subsubsection{Microprocessor project}
\begin{description}\parskip 0pt
\item[Person in charge:] Val�ry Lebret
\item[Level :] First year ENSEIRB-MATMECA - equivalent L3
\item[Volume :] 1 x 36 Hours
\item[Content:] This course aims to program MICROCHIP PIC microcontrollers, chosen for their ease of implementation due to their low complexity. 
After a presentation of this family of microcontrollers and their specificities, the activity begins by writing simple programs in assembler language to illustrate the operation of the microcontroller (coding and execution of instructions, access to registers, management of internal resources and inputs/outputs, etc.). 
An application card integrating a PIC16F84 serves as a support, with software development being carried out using the MPLABX integrated tool chain, which includes a simulator. 
Programming is then carried out in C language with the purpose of implementing a project (e. g. a quartz clock on an LCD display) using the PICDEM2 development board with a PIC16F877 target (more internal resources, possibility of debugging). The focus is on the limitations encountered on embedded systems when programming in C language (reduced memory space, limited computing power, etc.) as well as on interrupt management.
\item[Keywords:] Microcontroller programming, assembly language
\item[Personal participation :] Supervision of TPs
\end{description}


\noindent \subsubsection{Computer architecture}
\begin{description}\parskip 0pt
\item[Person in charge:] J�r�mie Crenne
\item[Level :] Second year ENSEIRB-MATMECA - equivalent M1
\item[Volume :] 1 x 16 Hours
\item[Content:] This course aims to strengthen knowledge by addressing more advanced techniques related to processors and memories. This course is based on the book by J.L. Hennessy and A. Patterson "Computer Architecture, a quantitative approach". RISC and CISC architectures, pipeline architectures, data and control hazards are discussed. The case of the MIPS processor is studied in detail to give students a concrete example of a processor architecture.
The purpose of this course is to enable students to understand the most sophisticated multi/many-core systems.
\item[Keywords:] Computer architecture, pipeline architectures, multithreading
\item[Personal Involvement :] Supervised works
\end{description}

\noindent \subsubsection{Microcomputer project}
\begin{description}\parskip 0pt
\item[Person in charge:] Yannick Bornat
\item[Level :] Second year ENSEIRB-MATMECA - equivalent M1
\item[Volume :] 1 x 42 Hours
\item[Content :] 
All practical works are performed on the AT91SAM7X256 microcontroller. This microcontroller has an ARM7 core, many peripherals and a vectorized and configurable interrupt system.
The objective of the course is to use the different hardware resources to design a minimalist operating system that meets the specific needs of microcontrollers for real-time tasks.
The concepts cover real-time aspects, communication, data integrity.
Students are placed in a situation where their only source of documentation is the manufacturer's technical documents in English.

\item[Keywords:] Microcontroller programming, C language, interrupt networks
\item[Personal participation:] Supervision of the practical works and the project
\end{description}

\noindent \subsubsection{Object oriented programming. C++ language}
\begin{description}\parskip 0pt
\item[Person in charge:] Bertrand Le Gal
\item[Level :] Second year ENSEIRB-MATMECA - equivalent M1
\item[Volume :] 1 x 15 Hours
\item[Content :] 
This course aims to provide students with the basics of object-oriented programming. The general concepts of object-oriented programming are being introduced. The C++ language is used to illustrate the concepts manipulated. All these concepts are used in a project to illustrate in a practical way the interest of this programming approach.
\item[Keywords:] Object oriented programming, C++
\item[Personal participation :] Supervision of practical works
\end{description}

\clearpage
%###################
